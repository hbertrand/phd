%%%%%%%%%%%%%%%%%%%%%%%%%%%%%%%%%%%%%%%%%%%%%%%%%%%%%%%%%%%%%%%%%%%%%%%%%%%%%%%%%%%%%%%%%%%%%%%
%                                          CHOLESKY                                           %
%%%%%%%%%%%%%%%%%%%%%%%%%%%%%%%%%%%%%%%%%%%%%%%%%%%%%%%%%%%%%%%%%%%%%%%%%%%%%%%%%%%%%%%%%%%%%%%
\chapter{Incremental Cholesky Decomposition Proofs}
\label{app:cholesky}

This appendix contains the derivations for obtaining the incremental Cholesky decomposition and its inverse that are used in Section~\ref{sec:cholesky}. For recall, the Cholesky decomposition and its inverse can be decomposed into blocks where one is the previous Cholesky decomposition. We obtain the formulas by developing this block decomposition.

Let us recall that the Cholesky decomposition $L_{(n)}$ of a positive definite matrix $K_{(n)}$ is a decomposition of the form:
\begin{equation}
	K_{(n)} = L_{(n)} L_{(n)}^T
\end{equation}
where $L_{(n)}$ is a lower triangular matrix. The decomposition is unique only if $K_{(n)}$ is positive definite. Since $K_{(n)}$ is a Gram matrix, it is always guaranteed to be positive semi-definite (i.e. $\forall v \in \mathbb{R}^n, \mkern10mu v^T K_{(n)} v \geq 0$). On top of that, if the rows and columns are unique (i.e. there are no duplicated data points), then it is positive definite.

%%%%%%%%%%%%%%%%%%%%%%%%%%%%%%%%%%%%%%%%%%%%%%%%%%%%%%%%%%%%%%%%%%%%%%%%%%%%%%%%%%%%%
\section[Formula for the Cholesky decomposition]{Formula for $L_{(n+k)}$}

When adding $k$ points to a Gram matrix of $n$ points, the block decomposition of the new Cholesky decomposition is:
\begin{alignat*}{2}
	&&L_{(n+k)} L_{(n+k)}^T &= K_{(n+k,n+k)} \\
	\Leftrightarrow\mkern40mu
	&&\begin{pmatrix}
    A & B \\
    C & D
  \end{pmatrix}
  \begin{pmatrix}
    A^T & C^T \\
    B^T & D^T
  \end{pmatrix} &= 
  \begin{pmatrix}
    K_{(n,n)} & K_{(k,n)}^T \\
    K_{(k,n)} & K_{(k,k)}
  \end{pmatrix}
\end{alignat*}
$B$ is obviously $0$ since a Cholesky decomposition is lower triangular.

By developing the block decomposition, we obtain the following equation for $A$:
\begin{equation*}
	A A^T + B B^T = A A^T = K_{(n,n)}
\end{equation*}
Since the Cholesky decomposition is unique, $A = L_{(n)}$. The equation for $C$ is also simple to solve:
\begin{alignat*}{2}
	&&C A^T + D B^T &= C L_{(n)}^T = K_{(k,n)} \\
	\Leftrightarrow\mkern40mu
	&&C &= K_{(k,n)} (L_{(n)}^T)^{-1}
\end{alignat*}
Solving for $D$ requires more work. Developing the block decomposition, we have:
\begin{alignat*}{2}
	&&C C^T + D D^T &= K_{(k,k)} \\
	\Leftrightarrow\mkern40mu
	&&D D^T &= K_{(k,k)} - K_{(k,n)} (K_{(n,n)})^{-1} K_{(k,n)}^T
\end{alignat*}
If we can show that $K_{(k,k)} - K_{(k,n)} (K_{(n,n)})^{-1} K_{(k,n)}^T$ is definite positive then $D$ is its unique Cholesky decomposition.

Let $w \in \mathbb{R}^{n+k} \quad \text{s.t.} \quad w = \begin{pmatrix}
    u \\
    v
  \end{pmatrix}, u \in \mathbb{R}^n, v \in \mathbb{R}^k$, we have $w^T K_{(n+k,n+k)} w \geq 0$ because $K_{(n+k,n+k)}$ is a Gram matrix. Using its block decomposition and developing it, we have:
\begin{alignat*}{2}
  &&\begin{pmatrix}
    u \\
    v
  \end{pmatrix} ^T
  \begin{pmatrix}
    K_{(n,n)} & K_{(k,n)}^T \\
    K_{(k,n)} & K_{(k,k)}
  \end{pmatrix}
  \begin{pmatrix}
    u \\
    v
  \end{pmatrix}
  &\geq 0 \\
  \Leftrightarrow\mkern40mu
  &&u^T K_{(n,n)} u + u^T K_{(k,n)}^T v +
  v^T K_{(k,n)} u + v^T K_{(k,k)} v
  &\geq 0 \\
  \Leftrightarrow\mkern40mu
  &&u^T K_{(n,n)} u + 
  2 u^T K_{(k,n)}^T v +
  v^T K_{(k,k)} v
  &\geq 0
\end{alignat*}
Using three successive change of variables, this equation becomes a second-order polynomial. The first change is $u = K_{(n,n)}^{-1} x$:
\begin{equation*}
  x^T (K_{(n,n)}^T)^{-1} x + 
  2 x^T (K_{(n,n)}^T)^{-1} K_{(k,n)}^T v +
  v^T K_{(k,k)} v
  \geq 0
\end{equation*}
The second change is $x = t y$ where $t$ is a scalar:
\begin{equation*}
  t^2 y^T (K_{(n,n)}^T)^{-1} y + 
  2 t y^T (K_{(n,n)}^T)^{-1} K_{(k,n)}^T v +
  v^T K_{(k,k)} v
  \geq 0
\end{equation*}
Since this is true for all $y \in \mathbb{R}^n$, this is true in particular when $y = K_{(k,n)}^T v$:
\begin{equation*}
  t^2 v^T K_{(k,n)} (K_{(n,n)}^T)^{-1} K_{(k,n)}^T v + 
  2 t v^T K_{(k,n)} (K_{(n,n)}^T)^{-1} K_{(k,n)}^T v +
  v^T K_{(k,k)} v
  \geq 0
\end{equation*}
The discriminant of this polynomial is negative or null because the polynomial is always positive or null:
\begin{align*}
  4 (v^T K_{(k,n)} (K_{(n,n)}^T)^{-1} K_{(k,n)}^T v)^2 -
  &4 (v^T K_{(k,n)} (K_{(n,n)}^T)^{-1} K_{(k,n)}^T v)
  (v^T K_{(k,k)} v)
  \leq 0 \\
  \Leftrightarrow\mkern20mu
  0 \leq (v^T K_{(k,n)} (K_{(n,n)}^T)^{-1} K_{(k,n)}^T v)^2
  &\leq 
  (v^T K_{(k,n)} (K_{(n,n)}^T)^{-1} K_{(k,n)}^T v)
  (v^T K_{(k,k)} v)
\end{align*}
$v^T K_{(k,k)} v \geq 0$ since $ K_{(k,k)}$ is a Gram matrix (i.e. positive semi-definite), meaning by necessity $v^T K_{(k,n)} (K_{(n,n)}^T)^{-1} K_{(k,n)}^T v \geq 0$ and $K_{(k,n)} (K_{(n,n)}^T)^{-1} K_{(k,n)}^T$ is positive semi-definite.

Moreover, $K_{(k,k)} \geq K_{(k,n)} (K_{(n,n)}^T)^{-1} K_{(k,n)}^T$, allowing us to conclude that $K_{(k,k)} - K_{(k,n)} (K_{(n,n)}^T)^{-1} K_{(k,n)}^T$ is positive semi-definite, and it is positive definite as long as none of the $k$ new combinations are duplicates of the $n$ previous combinations and $D D^T$ is its Cholesky decomposition.
\begin{equation*}
  D = cho(K_{(k,k)} - K_{(k,n)} (K_{(n,n)}^T)^{-1} K_{(k,n)}^T) = L_{(k)}
\end{equation*}
The final formula for the incremental Cholesky decomposition is:
\begin{equation}
  L_{(n+k)} = 
  \begin{pmatrix}
    L_{(n)} & 0 \\
    K_{(k,n)} (L_{(n)}^T)^{-1} & L_{(k)}
  \end{pmatrix}
\end{equation}
Since this formula requires the costly computation of $L_{(n)}^{-1}$, we would also like to find an incremental formula for it.

%%%%%%%%%%%%%%%%%%%%%%%%%%%%%%%%%%%%%%%%%%%%%%%%%%%%%%%%%%%%%%%%%%%%%%%%%%%%%%%%%%%%%
\section[Formula for the inverse Cholesky decomposition]{Formula for $L_{(n+k)}^{-1}$}

A standard expression for the inversion of a block matrix is:
\begin{equation*}
  \begin{pmatrix}
    A & B \\
    C & D
  \end{pmatrix}^{-1} = 
  \begin{pmatrix}
    A^{-1} + A^{-1} B (D - C A^{-1} B)^{-1} C A^{-1} & - A^{-1} B (D - C A^{-1} B)^{-1} \\
    - (D - C A^{-1} B)^{-1} C A^{-1} & (D - C A^{-1} B)^{-1}
  \end{pmatrix}
\end{equation*}
Simplifying with $L_{(n+k)}$ found previously, we have:
\begin{equation}
  L_{(n+k)}^{-1} =
  \begin{pmatrix}
    L_{(n)} & 0 \\
    K_{(k,n)} (L_{(n)}^T)^{-1} & L_{(k)}
  \end{pmatrix}^{-1} = 
  \begin{pmatrix}
    L_{(n)}^{-1} & 0 \\
    - L_{(k)}^{-1} K_{(k,n)} (L_{(n)}^T)^{-1} L_{(n)}^{-1} & L_{(k)}^{-1}
  \end{pmatrix}
\end{equation}
